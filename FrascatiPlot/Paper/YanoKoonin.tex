\RequirePackage{ifpdf}

%\documentclass[a4paper,11pt]{scrartcl} % scrartcl scrreprt
\documentclass[a4paper,11pt]{scrartcl} % scrartcl scrreprt

\usepackage[margin=1.5cm]{geometry}


\ifpdf
  \usepackage[pdftex]{graphicx}
\else
  \usepackage[dvips]{graphicx}
\fi


\usepackage[utf8]{inputenc}


\usepackage[T1]{fontenc}


\usepackage{amsmath}
\usepackage[amssymb]{SIunits}
\usepackage{hyperref}




%\usepackage{lineno}
%\linenumbers


%
% DOCUMENT STARTS HERE
%


\begin{document}
\begin{center}
{\large {\bf HBT volume in Bertsch-Pratt and Yano-Koonin-Podgoretskii parameterisation for comparison with coalescence models}}
\bigskip

A. Kalweit, 12th of June 2015
\end{center}

%\smallskip

%
% Introduction
%

\paragraph{Introduction.}

\noindent The results of HBT analyses are typically presented in either the Bertsch-Pratt ($R_{o}$, $R_{s}$, $R_{l}$) or the Yano-Koonin-Podgoretskii ($R_{\perp}$, $R_{0}$, $R_{\parallel}$) parameterization. The ALICE HBT results \cite{Aamodt:2011mr} are given in the Bertsch-Pratt convention. However, for a comparison with coalescence models for deuteron production, the volume expressed in terms of the Yano-Koonin-Podgoretskii (YKP) is needed, because the coalescence parameter $B_{2}$ is given by

\begin{equation}
 B_{2} = \frac{3\pi^{3/2} \langle C_{\rm d}\rangle}{2m_{\rm T} R^{2}_{\perp}(m_{\rm T}) R_{\parallel}(m_{\rm T})}  \; ,
\label{B2nCd}
\end{equation}

\noindent as derived in \cite{Scheibl:1998tk}. We therefore need to express $R^{2}_{\perp}(m_{\rm T}) R_{\parallel}(m_{\rm T})$ in terms of the Bertsch-Pratt (BP) radii.


\paragraph{Transformation.} The transformation between the two parameterisations is best presented in~\cite{Wiedemann:1999qn} in the equations (W 3.48) to (W 3.52)\footnote{The equations in~\cite{Wiedemann:1999qn} are denoted as (W...) in order to distinguish them from the equations presented in this writeup.}:

\begin{eqnarray}
R_{s}^{2}    &=& R_{\perp}^{2}  \;, \\
R_{diff}^{2} &=& R_{o}^{2} -  R_{s}^{2} = \beta_{\perp}^2 \gamma^2 (R_{0}^2 + v^2R_{\parallel}^2) \;, \\
R_{l}^{2}     &=& (1 - \beta_l^2) R_{\parallel}^2 + \gamma^2 (\beta_l - v)^2 (R_{0}^2 + v^2R_{\parallel}^2) \;, \\
R_{ol}^{2}   &=& \beta_{\perp} (-\beta_l R_{\parallel}^{2} +  \gamma^2 (\beta_l - v)(R_{0}^2 + v^2R_{\parallel}^2)) \,.
\end{eqnarray}

\noindent From this we immediately identify that we can replace $R_{\perp}^2$ with $R_{s}^{2}$ in Equation~(\ref{B2nCd}). However, an expression for $R_{\parallel}$ is still needed. The inversion of the above equations is also explained in~\cite{Wiedemann:1999qn}. Following this nomenclature (W 3.52-3.53), $R_{\parallel}^{2}$ can thus be expressed as 

%\begin{eqnarray}
% R_{\parallel}^{2} &=& B  - v \cdot C \;, \\
%                          &=& R_l^2 - 2{\beta_l \over \beta_\perp}  R_{ol}^2 + {\beta_l^2 \over \beta_\perp^2} R_{diff}^2 \\
%                           &  &  - v \cdot \Bigl(-{1 \over \beta_\perp } R_{ol}^2 + {\beta_l \over \beta_\perp^2} R_{diff}^2  \Bigr) \; .
%\end{eqnarray}
\begin{eqnarray}
R_{\parallel}^{2} &=& B  - v \cdot C \;, \\
                          &=& R_l^2 - 2{\beta_l \over \beta_\perp}  R_{ol}^2 + {\beta_l^2 \over \beta_\perp^2} R_{diff}^2 
                          - v \cdot \Bigl(-{1 \over \beta_\perp } R_{ol}^2 + {\beta_l \over \beta_\perp^2} R_{diff}^2  \Bigr) \; .
\end{eqnarray}


\noindent As it turns out all corrections which are subtracted from $R_{l}^{2}$ can be neglected. 

\begin{itemize}
	\item As also stated in \cite{Aamodt:2011mr}, $R_{ol} \approx 0$ for symmetric collision systems. Since this is the case in Pb--Pb collisions (for p--Pb collisions one would probably need to be careful).
	\item The remaining terms are all proportional to $\beta_l$. Since the ALICE HBT results are always presented in the LCMS (longitudinally co-moving system) frame, $\beta_l = 0$ by the definition of the rest frame.

\end{itemize}

\noindent We can thus summarise our findings as: $R_{\perp}=R_{s}$ and $R_{\parallel}=R_{l}$ for symmetric collision systems and results presented in the LCMS.


\bigskip

\begin{thebibliography}{9} % !!!!!!!!!!!!!!!!!!!!!!!!!!!!!!!!!!!!!!!!!!!! ORDER THEM !!!!!!!!!!!!!!!!!!!!!!!!!!!!!

 \bibitem{Aamodt:2011mr}
  K.~Aamodt {\it et al.}  [ALICE Collaboration],
  %``Two-pion Bose-Einstein correlations in central Pb-Pb collisions at $sqrt(s_NN)$ = 2.76 TeV,''
  Phys.\ Lett.\ B {\bf 696} (2011) 328
  [arXiv:1012.4035 [nucl-ex]].
  %%CITATION = ARXIV:1012.4035;%%
  %170 citations counted in INSPIRE as of 12 Jun 2015

 \bibitem{Wiedemann:1999qn}
  U.~A.~Wiedemann and U.~W.~Heinz,
  %``Particle interferometry for relativistic heavy ion collisions,''
  Phys.\ Rept.\  {\bf 319} (1999) 145
  [nucl-th/9901094].
  %%CITATION = NUCL-TH/9901094;%%
  %336 citations counted in INSPIRE as of 12 juin 2015

 \bibitem{Scheibl:1998tk}
  R.~Scheibl and U.~W.~Heinz,
  %``Coalescence and flow in ultrarelativistic heavy ion collisions,''
  Phys.\ Rev.\ C {\bf 59} (1999) 1585
  [nucl-th/9809092].
  %%CITATION = NUCL-TH/9809092;%%
  %117 citations counted in INSPIRE as of 12 juin 2015



\end{thebibliography}


\end{document}




%%
%% End of file.
