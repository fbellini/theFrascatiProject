\label{appendix:integration}

In this appendix, we show the calculation of the RMS of a gaussian wave-function with the example of $A=4$. We start from the wave-function

\begin{equation}\label{eq:SatoWave}
 \phi_{A=4} = C_{4} \exp\Bigl(-{1\over2} \nu_{4} \sum_{i=1}^{4}(\vec{x}_{i} - \vec{X})^2 \Bigr) \; .
\end{equation}

\noindent Then we introduce the Jacobi variables which are given by 
\begin{align}
	\vec{X}    &= {1 \over 4} (\vec{x_1} + \vec{x_2} + \vec{x_3} + \vec{x_4}) \\
	\vec{r_1} &= \vec{x_2} - \vec{x_1}\\
	\vec{r_2} &= \vec{x_3} - {1 \over 2}(\vec{x_1} + \vec{x_2})\\
	\vec{r_3} &= \vec{x_4} - {1 \over 3}(\vec{x_1} + \vec{x_2} + \vec{x_3}) \; .
\end{align}

\noindent In this parameterisation we can express the squared distance of each nucleon with respect to the centre-of-mass as

\begin{equation}
 \sum_{i=1}^{4} \bigl(\vec{x_i} - \vec{X} \bigr)^2 = {1\over 4} \bigl(2\vec{r_1} + {8\over 3}\vec{r_2} + 3\vec{r_3}\bigr)  \; .
\end{equation}

\noindent For the following integral calculations we determine the Jacobian as

\begin{align}
	\mathrm{d}R\mathrm{d}r_{1}\mathrm{d}r_{2}\mathrm{d}r_{3} &= \vert\det {\partial(R,r_{1},r_{2},r_{3}) \over \partial(x_{1},x_{2},x_{3},x_{4})}\vert \; \mathrm{d}x_{1}\mathrm{d}x_{2}\mathrm{d}x_{3}\mathrm{d}x_{4} \\
		&= \begin{vmatrix} {\partial R \over \partial x_{1}} & {\partial r_{1} \over \partial x_{1}} & {\partial r_{2} \over \partial x_{1}}& {\partial r_{3} \over \partial x_{1}} \\ 
		{\partial R \over \partial x_{2}} & {\partial r_{1} \over \partial x_{2}} & {\partial r_{2} \over \partial x_{2}}& {\partial r_{3} \over \partial x_{2}} \\ 
		{\partial R \over \partial x_{3}} & {\partial r_{1} \over \partial x_{3}} & {\partial r_{2} \over \partial x_{3}}& {\partial r_{3} \over \partial x_{3}} \\ 
		{\partial R \over \partial x_{4}} & {\partial r_{1} \over \partial x_{4}} & {\partial r_{2} \over \partial x_{4}}& {\partial r_{3} \over \partial x_{4}} \end{vmatrix} 
		\mathrm{d}x_{1}\mathrm{d}x_{2}\mathrm{d}x_{3}\mathrm{d}x_{4} \\
		&=  \begin{vmatrix} {1\over 4} & -1 & -{1\over 2} & -{1\over 3}  \\
			{1\over 4} & 1 & -{1\over 2} & -{1\over 3}  \\
			{1\over 4} & 0 & 1 & -{1\over 3}  \\
			{1\over 4} & 0 & 0 & 1
			 \end{vmatrix} \mathrm{d}x_{1}\mathrm{d}x_{2}\mathrm{d}x_{3}\mathrm{d}x_{4} \\
		&= 1 \cdot \mathrm{d}x_{1}\mathrm{d}x_{2}\mathrm{d}x_{3}\mathrm{d}x_{4} \; .
\end{align}

\noindent From the requirement $\int|\phi|^2\mathrm{d}\vec{r}_{1}\mathrm{d}\vec{r}_{2}\mathrm{d}\vec{r}_{3} = 1$, we thus obtain for the normalisation constant $C_4$: 

\begin{align}
	{1\over C_{4}^{2}} &= \int \exp^{2}\Bigl(-{1\over 2} \nu_{4} ({1\over 2}r_{1}^{2} + {2\over 3} r_{2}^{2} + {3\over 4}r_{3}^{2}) \Bigr) \mathrm{d}\vec{r}_{1}\mathrm{d}\vec{r}_{2}\mathrm{d}\vec{r}_{3}
\end{align}

\noindent where the integration in spherical coordinates is performed with $\mathrm{d}\vec{r}_{1} = 4\pi r_{1}^{2}\mathrm{d}r_{1},  \mathrm{d}\vec{r}_{2} = \ldots$ which gives

\begin{align}
	{1\over C_{4}^{2}} &= (4\pi)^{3} \underbrace{\int_{0}^{\infty} r_{1}^{2}\exp\Bigl(-{1\over 2} \nu_{4}  r_{1}^{2}  \Bigr)  \mathrm{d}r_{1}}_{{\sqrt{\pi} \over 4} {1 \over ({1\over 2} \nu_{4})^{3/2}}  }
							\underbrace{\int_{0}^{\infty} r_{2}^{2}\exp\Bigl(-{2\over 3} \nu_{4}  r_{2}^{2}  \Bigr)  \mathrm{d}r_{2}}_{{\sqrt{\pi} \over 4} {1 \over ({2\over 3} \nu_{4})^{3/2}}  }
							\underbrace{\int_{0}^{\infty} r_{3}^{2}\exp\Bigl(-{3\over 4} \nu_{4}  r_{3}^{2}  \Bigr)  \mathrm{d}r_{3}}_{{\sqrt{\pi} \over 4} {1 \over ({3\over 4} \nu_{4})^{3/2}}  } \\
		&= \bigl({\pi \over \nu_{A}}\bigr)^{9/2} 4^{3/2} \\
	\Rightarrow C_{4} &= \Bigl({\nu_{A}^{3} \over 4 \pi^{3}}\Bigr)^{3/4} \;.
\end{align}

\noindent For the rms, we obtain accordingly:


\begin{align}
	\lambda_{4}^{2} &= {1 \over 4} \langle \phi \vert   \sum_{i=1}^{4}(\vec{x}_{i} - \vec{X})^2 \vert \phi \rangle \\
	&= \int {1\over 4}  \sum_{i=1}^{4}(\vec{x}_{i} - \vec{X})^2 |\phi|^2 \mathrm{d}\vec{x}_{1}\mathrm{d}\vec{x}_{2}\mathrm{d}\vec{x}_{3}\mathrm{d}\vec{x}_{4} \\
	 &= {C_4^2 \over 4}  \int \bigl({1\over 2}r_{1}^{2} + {2\over 3} r_{2}^{2} + {3\over 4}r_{3}^{2} \bigr) \exp^{2}\Bigl(-{1\over 2} \nu_{4} ({1\over 2}r_{1}^{2} + {2\over 3} r_{2}^{2} + {3\over 4}r_{3}^{2}) \Bigr) \mathrm{d}\vec{r}_{1}\mathrm{d}\vec{r}_{2}\mathrm{d}\vec{r}_{3} \\
	&= {C_4^2 \over 4} \Bigl(\underbrace{{1\over 2} \int r_{1}^2 \exp\bigl(-{1\over2} \nu_A r_{1}^{2}\bigr) \exp\bigl(-{2\over3} \nu_A r_{2}^{2}\bigr) \exp\bigl(-{3\over4} \nu_A r_{3}^{2}\bigr) \mathrm{d}\vec{r}_{1}\mathrm{d}\vec{r}_{2}\mathrm{d}\vec{r}_{3}}_{\mathrm{I}} \\
	&\;\;+ \underbrace{{2\over 3} \int r_{2}^2 \exp\bigl(-{1\over2} \nu_A r_{1}^{2}\bigr) \exp\bigl(-{2\over3} \nu_A r_{2}^{2}\bigr) \exp\bigl(-{3\over4} \nu_A r_{3}^{2}\bigr) \mathrm{d}\vec{r}_{1}\mathrm{d}\vec{r}_{2}\mathrm{d}\vec{r}_{3}}_{\mathrm{II}} \\
	&\;\;+ \underbrace{{3\over 4} \int r_{3}^2 \exp\bigl(-{1\over2} \nu_A r_{1}^{2}\bigr) \exp\bigl(-{2\over3} \nu_A r_{2}^{2}\bigr) \exp\bigl(-{3\over4} \nu_A r_{3}^{2}\bigr) \mathrm{d}\vec{r}_{1}\mathrm{d}\vec{r}_{2}\mathrm{d}\vec{r}_{3}}_{\mathrm{III}}
				 \Bigr)
\end{align}

\noindent The first summand I can be calculated with $\mathrm{d}\vec{r}_{1}\mathrm{d}\vec{r}_{2}\mathrm{d}\vec{r}_{3} = (4\pi)^3 r_1^2 r_2^2 r_3^2 \mathrm{d}r_{1}\mathrm{d}r_{2}\mathrm{d}r_{3}$ as

\begin{align}
 \mathrm{I} &= {1\over 2} (4\pi)^3 \cdot \int_{0}^{\infty} r_{1}^2 \exp\bigl(-{1\over2} \nu_A r_{1}^{2}\bigr)\mathrm{d}r_{1} \cdot \int_{0}^{\infty}\exp\bigl(-{2\over3} \nu_A r_{2}^{2}\bigr) \mathrm{d}r_{2} \cdot \int_{0}^{\infty} \exp\bigl(-{3\over4} \nu_A r_{3}^{2}\bigr) \mathrm{d}r_{3} \\
 	&= {1\over 2} (4\pi)^3 \cdot {3\sqrt{\pi} \over 8} {1 \over ({1\over 2} \nu_{A})^{5/2}} \cdot {\sqrt{\pi} \over 4} {1 \over ({2\over 3} \nu_{A})^{3/2}} \cdot {\sqrt{\pi} \over 4} {1 \over ({3\over 4} \nu_{A})^{3/2}} \;.
\end{align}

\noindent The other two summands II and III are evaluated accordingly and we thus obtain for the rms:

\begin{align}
	\lambda_{4}^{2} &= {C_4^2 \over 4} (\mathrm{I} + \mathrm{II} + \mathrm{III} ) \\
	&= {C_4^2 \over 4} (4\pi)^3 {3\pi^{3/2} \over 8 \cdot 4^2 \cdot \nu_A^{11/2}} \underbrace{\Bigl({{1\over2} \over ({1\over2})^{5/2}({2\over3})^{3/2}({3\over4})^{3/2} } +
	{{2\over3} \over ({1\over2})^{3/2}({2\over3})^{5/2}({3\over4})^{3/2} } + {{3\over4} \over ({1\over2})^{3/2}({2\over3})^{3/2}({3\over4})^{5/2} } \Bigr)}_{= 24} \\
	&= {9 \over 8 \nu_A} \;.
\end{align}
