\label{appendix:YKP}

The results of HBT analyses are typically presented in either the Bertsch-Pratt ($R_{out}$, $R_{side}$, $R_{long}$) or the Yano-Koonin-Podgoretskii ($R_{\perp}$, $R_{0}$, $R_{\parallel}$) parameterization. The ALICE HBT results \cite{Aamodt:2011mr, Adam:2015vna} are given in the Bertsch-Pratt convention, whereas the coalescence parameter is derived in \cite{Scheibl:1998tk} by expressing the dependence on the volume in terms of the Yano-Koonin-Podgoretskii (YKP) parameterisation. 
The transformation between the two parameterisations is best presented in~\cite{Wiedemann:1999qn} in the equations (W 3.48) to (W 3.52)\footnote{The equations in~\cite{Wiedemann:1999qn} are denoted as (W...) in order to distinguish them from the equations presented in this paper.}:

\begin{eqnarray}
R_{side}^{2}    &=& R_{\perp}^{2}  \;, \\
R_{diff}^{2} &=& R_{out}^{2} -  R_{side}^{2} = \beta_{\perp}^2 \gamma^2 (R_{0}^2 + v^2R_{\parallel}^2) \;, \\
R_{long}^{2}     &=& (1 - \beta_l^2) R_{\parallel}^2 + \gamma^2 (\beta_l - v)^2 (R_{0}^2 + v^2R_{\parallel}^2) \;, \\
R_{ol}^{2}   &=& \beta_{\perp} (-\beta_l R_{\parallel}^{2} +  \gamma^2 (\beta_l - v)(R_{0}^2 + v^2R_{\parallel}^2)) \,.
\end{eqnarray}
%
We immediately identify that $R_{\perp}^2$ can be identified with $R_{side}^{2}$. Following the reasoning and the nomenclature in~\cite{Wiedemann:1999qn} (W 3.52-3.53), the above equations can be inverted and $R_{\parallel}^{2}$ can be expressed as 
%
\begin{eqnarray}
R_{\parallel}^{2} &=& B  - v \cdot C \;, \\
                          &=& R_{long}^2 - 2{\beta_l \over \beta_\perp}  R_{ol}^2 + {\beta_l^2 \over \beta_\perp^2} R_{diff}^2 
                          - v \cdot \Bigl(-{1 \over \beta_\perp } R_{ol}^2 + {\beta_l \over \beta_\perp^2} R_{diff}^2  \Bigr) \; .
\end{eqnarray}
%
As it turns out all corrections which are subtracted from $R_{long}^{2}$ can be neglected. First, we notice that the cross term $R_{ol}$ vanishes if the measured fireball is longitudinally boost-invariant, which is a valid approximation for the rapidity ranges studied here. The remaining terms are all proportional to $\beta_l$. By definition of the rest frame for a longitudinally-boosted invariant system, $\beta_l = 0$. In summary, we can consider $R_{\perp} = R_{side}$ and $R_{\parallel}=R_{long}$ for the present study.
