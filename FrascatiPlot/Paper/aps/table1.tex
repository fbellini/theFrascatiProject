\begin{table*}[htb]

\centering

\begin{ruledtabular}
\begin{tabular}{cccccccc}\\[-2ex]
Mass number & Nucleus &  Composition  & $B_{E}$ (MeV)   &  $J_{A}$ & \rmsradius$^{meas}$~(fm) &  $r_{A}$ (fm) & Refs. 
\\[0.5ex] \hline \\[-2ex]
      A = 2                     & d                                    & pn                                  &   2.224575 (9)     &     1   & 2.1413 $\pm$ 0.0025      &  3.2    &   \cite{VanDerLeun:1982bhg,Mohr:2015ccw}     \\[0.5ex]  \hline \\[-2ex]
      A = 3  & \tritium 	                  & pnn                               &    8.4817986 (20) & 1/2   &  1.755  $\pm$ 0.086        &  2.15   &   \cite{Purcell:2015gtm}           \\
                                   & \hethree                         & ppn                                &   7.7180428  (23) & 1/2  & 1.959 $\pm$  0.030         &   2.48  &   \cite{Purcell:2015gtm} \\
                                   & \hthreelambda               & p$\Lambda$n                &    0.13 $\pm$ 0.05 & 1/2  &  4.9 --  10.0                    &  6.8 -- 14.1 & \cite{Davis:2005mb,Nemura:1999qp} \\[0.5ex]  \hline \\[-2ex]
                                   A = 4  & \hefour                          & ppnn                              &    28.29566   (20)  &      0  &  1.6755 $\pm$ 0.0028  &  1.9  & \cite{1674-1137-41-3-030003,Angeli:2013epw} \\
                                   & \hfourlambda                & p$\Lambda$nn              &  2.04 $\pm$0.04   &   0   &    2.0 -- 3.8             & 2.4 -- 4.9  & \cite{Davis:2005mb,Nemura:1999qp} \\
                                   & \hfourtwolambda          &  p$\Lambda\Lambda$n &   0.39 -- 0.51         &    1 \cite{Nemura:1999qp}    &    4.2 -- 7.1          & 5.5 -- 9.4  &   \cite{Nemura:1999qp} \\
                                   & \hefourlambda              & pp$\Lambda$n              &  2.39 $\pm$ 0.03  &    0   &    2.0 -- 3.8            & 2.4 -- 4.9  & \cite{Davis:2005mb,Nemura:1999qp}\\[0.5ex] 
                                   
\end{tabular}
\end{ruledtabular}

\caption{Properties of nuclei and hyper-nuclei with mass number $A \leq 4$. $B_{E}$ is the binding energy in MeV, $J_{A}$ is the spin. The size of the nucleus is given in terms of the (charge) rms radius of the wave-function, \rmsradius. The size parameter of the wave-function of the harmonic oscillator potential,  $r_{A}$, is chosen such that the measured/expected rms,  \rmsradius$^{meas}$~(fm), is approximately reproduced. Please note that the proton rms charge radius $\lambda_{p} = 0.879(8)$ fm \cite{bernauer10} is subtracted quadratically from $\rmsradius^{meas}$ of the nucleus $\rmsradius = \sqrt{(\rmsradius^{meas})^2 - {\lambda_{p}^2}}$ to account for the finite extension of the constituents. Implicitly we assume here that $\lambda_{\Lambda}\approx \lambda_{n}\approx \lambda_{p}$.}
 %References are given in the last column.}
%. The spin of \hfourtwolambda\ is discussed in the text of~\cite{Nemura:1999qp}.
\label{tab:nucleusradii}
\end{table*}
