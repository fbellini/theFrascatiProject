\begin{table*}[htb]

\centering

\begin{ruledtabular}
\begin{tabular}{cccccccc}\\[-2ex]
Mass number & Nucleus &  Composition  & Binding energy (MeV)   &  Spin & \rmsradius$^{meas}$~(fm) &  $r_{A}$ (fm) & Refs. 
\\[0.5ex] \hline \\[-2ex]
      A = 2                     & d                                    & pn                                  &   2.224575 (9)     &     1   & 2.1413 $\pm$ 0.0025      &  3.2    &   \cite{VanDerLeun:1982bhg,Mohr:2015ccw}     \\[0.5ex]  \hline \\[-2ex]
      A = 3  & \tritium 	                  & pnn                               &    8.4817986 (20) & 1/2   &  1.755  $\pm$ 0.086        &  2.15   &   \cite{Purcell:2015gtm}           \\
                                   & \hethree                         & ppn                                &   7.7180428  (23) & 1/2  & 1.959 $\pm$  0.030         &   2.48  &   \cite{Purcell:2015gtm} \\
                                   & \hthreelambda               & p$\Lambda$n                &    0.13 $\pm$ 0.05 & 1/2  &  4.9 --  10.0                    &  6.8 -- 14.1 & \cite{Davis:2005mb,Nemura:1999qp} \\[0.5ex]                                    
\end{tabular}
\end{ruledtabular}

\caption{Properties of nuclei and hyper-nuclei with mass number $A \leq 3$. The nucleus size is given in terms of the (charge) rms radius of the wave-function, \rmsradius. The size parameter of the wave-function of the harmonic oscillator potential,  $r_{A}$, is chosen such that the measured/expected rms,  \rmsradius$^{meas}$~(fm), is approximately reproduced. The proton rms charge radius $\lambda_{p} = 0.879(8)$ fm \cite{bernauer10} is subtracted from $\rmsradius^{meas}$  according to $\rmsradius = \sqrt{(\rmsradius^{meas})^2 - {\lambda_{p}^2}}$ to account for the finite extension of the constituents. We assume $\lambda_{\Lambda}\approx \lambda_{n}\approx \lambda_{p}$.}
\label{tab:nucleusradii}
\end{table*}
