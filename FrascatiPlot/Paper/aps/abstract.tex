
We present a detailed comparison of coalescence and thermal-statistical models for the production of (anti-)(hyper-)nuclei in high-energy collisions. For the first time, such a study is carried out as a function of the size of the object relative to the size of the particle emitting source. Our study reveals large differences between the two scenarios for the production of objects with extended wave-functions. While both models give similar predictions and show similar agreement with experimental data for (anti-)deuterons and (anti-)\hethree\ nuclei, they largely differ in their description of (anti-)hyper-triton production.
We propose to address experimentally the comparison of the production models by measuring the coalescence parameter systematically for different (anti-)(hyper-)nuclei in different collision systems and differentially in multiplicity. 
Such measurements are feasible with the current and upgraded LHC experiments. 
Our findings highlight the unique potential of ultra-relativistic heavy-ion collisions as a laboratory to clarify the internal structure of exotic QCD objects and can serve as a basis for more refined calculations in the future.
