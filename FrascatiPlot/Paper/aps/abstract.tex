(Anti-)(hyper-)nuclei are unique probes of the medium created in proton-proton, proton-Pb, and Pb--Pb collisions at the energies available at the Large Hadron Collider (LHC). Their production is typically discussed within the framework of coalescence and thermal-statistical production models. While it is often argued that these two approaches are not distinguishable, we present a detailed study of both theories which reveals large differences between the two scenarios for the production of objects with extended wave-functions. 
We propose for the first time a study which is experimentally feasible with the current and upgraded LHC experiments to distinguish between the two production scenarios.
Both models give similar predictions and show similar agreement with experimental data for (anti-)deuterons and (anti-)\hethree\ nuclei, but they largely differ in their description of (anti-)hyper-triton production.
The thermal-statistical model is found to be in agreement with results in central Pb--Pb collisions even though fragile objects such as light (anti-)(hyper-)nuclei should be destroyed in hadronic interactions after the chemical freeze-out of the system. Our findings highlight the unique potential of ultra-relativistic heavy-ion collisions as a laboratory to clarify the internal structure of exotic QCD objects and can serve as a basis for more refined calculations in the future.
