\RequirePackage{ifpdf}

%\documentclass[a4paper,11pt]{scrartcl} % scrartcl scrreprt
\documentclass[a4paper,11pt]{scrartcl} % scrartcl scrreprt

\usepackage[margin=2.cm]{geometry}


\ifpdf
  \usepackage[pdftex]{graphicx}
\else
  \usepackage[dvips]{graphicx}
\fi

%\usepackage[utf8]{inputenc}
\usepackage[T1]{fontenc}

\usepackage{amsmath}
\usepackage[amssymb]{SIunits}
\usepackage{hyperref}

\usepackage{lineno}
\linenumbers

\usepackage{xcolor}

%
% DOCUMENT STARTS HERE
%
% $Id: commands.tex 909 2013-06-03 14:10:59Z rpreghen $
% Adaptation of a file originally by Roberto Preghenella (thanks!) 
%==========================================================%
\newcommand{\btwo}{\ensuremath{B_{2}}}
\newcommand{\bthree}{\ensuremath{B_{3}}}
\newcommand{\bA}{\ensuremath{B_{A}}}
\newcommand{\bthreeLambda}{$B_{3, \Lambda}$}

%nuclei and hypernuclei
\newcommand{\tritium}{\ensuremath{{}^{3}\mathrm{H}}}
\newcommand{\hethree}{\ensuremath{{}^{3}\mathrm{He}}}
\newcommand{\hefour}{\ensuremath{{}^{4}\mathrm{He}}}
\newcommand{\hthreelambda}{\ensuremath{{}^{3}_{\Lambda}\mathrm{H}}}
\newcommand{\hfourlambda}{\ensuremath{{}^{4}_{\Lambda}\mathrm{H}}}
\newcommand{\hfourtwolambda}{\ensuremath{{}^{4}_{\Lambda\Lambda}\mathrm{H}}}
\newcommand{\hefourlambda}{\ensuremath{{}^{4}_{\Lambda}\mathrm{He}}}

\newcommand{\antitritium}{\ensuremath{{}^{3}\overline{\mathrm{He}}}}
\newcommand{\antihethree}{\ensuremath{{}^{3}\overline{\mathrm{He}}}}
\newcommand{\antihefour}{\ensuremath{${}^{4}$\overline{\mathrm{He}}}}
\newcommand{\antihthreelambda}{\ensuremath{{}^{3}_{\Lambda}\overline{\mathrm{He}}}}
\newcommand{\antihfourlambda}{\ensuremath{{}^{4}_{\Lambda}\overline{\mathrm{H}}}}
\newcommand{\antihefourlambda}{\ensuremath{{}^{4}_{\Lambda}\overline{\mathrm{He}}}}
\newcommand{\antihfourtwolambda}{\ensuremath{{}^{4}_{\Lambda\Lambda}\overline{\mathrm{H}}}}

\newcommand{\dradius}{\ensuremath{r_{d}}}
\newcommand{\rperp}{\ensuremath{R_{\perp}}}
\newcommand{\rpar}{\ensuremath{R_{\|}}}



%==========================================================%
\newcommand{\Om}{$\Omega^-$}
\newcommand{\Mo}{$\overline{\Omega}^+$}
\newcommand{\X}{$\Xi^-$}
\newcommand{\Ix}{$\overline{\Xi}^+$}
\newcommand{\Xis}{$\Xi^{\pm}$}
\newcommand{\Oms}{$\Omega^{\pm}$}
\newcommand{\meanpt}{$\langle p_\mathrm{t}\rangle$}
\newcommand{\nineH}{$\sqrt{s}~=~0.9$~TeV}
\newcommand{\seven}{$\sqrt{s}~=~7$~TeV}
\newcommand{\twoH}{$\sqrt{s}~=~0.2$~TeV}
\newcommand{\dndy}{d$N$/d$y$}
\newcommand{\LT}{L{\'e}vy-Tsallis}
\newcommand{\GeVc}{GeV/$c$}
\newcommand{\MeVc}{MeV/$c$}
\newcommand{\GeVcs}{GeV/$c$ }
\newcommand{\MeVcs}{MeV/$c$ }
\newcommand{\GeVmass}{GeV/$c^2$}
\newcommand{\MeVmass}{MeV/$c^2$}
\newcommand{\allpart}{\kzero, \lmb, \almb, \X, \Ix, \Om and \Mo}

\newcommand{\ITS}          {\rm{ITS }}
\newcommand{\TOF}          {\rm{TOF }}
\newcommand{\ZDC}          {\rm{ZDC }}
\newcommand{\ZDCs}         {\rm{ZDCs }}
\newcommand{\ZNA}          {\rm{ZNA }}
\newcommand{\ZNC}          {\rm{ZNC }}
\newcommand{\SPD}          {\rm{SPD }}
\newcommand{\SDD}          {\rm{SDD }}
\newcommand{\SSD}          {\rm{SSD }}
\newcommand{\TPC}          {\rm{TPC }}
\newcommand{\VZERO}        {\rm{VZERO }}
\newcommand{\VZEROA}       {\rm{VZERO-A }}
\newcommand{\VZEROC}       {\rm{VZERO-C }}
\newcommand{\pip}          {\ensuremath{\pi^{+}}}
\newcommand{\pim}          {\ensuremath{\pi^{-}}}
\newcommand{\pipm}          {\ensuremath{\pi^{\pm}}}
\newcommand{\kap}          {\ensuremath{\mathrm{K}^{+}}}
\newcommand{\kam}          {\ensuremath{\mathrm{K}^{-}}}
\newcommand{\kapm}          {\ensuremath{\mathrm{K}^{\pm}}}
\newcommand{\p}               {$\rm p$}
\newcommand{\pbar}         {$\rm\overline{p}$}
\newcommand{\kzero}        {\ensuremath{{\rm K}^{0}_{S}}}
\newcommand{\kstar}        {\ensuremath{{\rm K}^{*0}}}
\newcommand{\lmb}          {\ensuremath{\Lambda}}
\newcommand{\almb}         {\ensuremath{\overline{\Lambda}}}
%this is another analysis...
%\newcommand{\allpart}      {$\pi^{\pm}$, K$^{\pm}$, \kzero, p(\pbar) and \lmb(\almb)}
%\newcommand{\degree}       {$^{\rm o}$}
\newcommand{\dg}           {\mbox{$^\circ$}}
\newcommand{\dedx}         {\ensuremath{\mathrm{d}E/\mathrm{d}x}}
\newcommand{\pp}           {pp}
\newcommand{\ppbar}        {\mbox{$\mathrm {p\overline{p}}$}}
\newcommand{\PbPb}         {\mbox{Pb--Pb}}
\newcommand{\pPb}          {\mbox{p--Pb}}
\newcommand{\AuAu}         {\mbox{Au--Au}}
\newcommand{\pseudorap}    {\mbox{$\left | \eta \right | $}}
\newcommand{\dNdeta}       {\ensuremath{\mathrm{d}N_\mathrm{ch}/\mathrm{d}\eta}}

\newcommand{\avdNdeta}       {\ensuremath{\left<\mathrm{d}N_\mathrm{ch}/\mathrm{d}\eta\right>}}
\newcommand{\dNchdy}         {\ensuremath{\mathrm{d}N_\mathrm{ch}/\mathrm{d}y}}
\newcommand{\dNdy}         {\ensuremath{\mathrm{d}N/\mathrm{d}y} }
\newcommand{\dNdyst}       {\ensuremath{\sqrt{\frac{dN_\pi/dy}{s_T}}}}
\newcommand{\dNdetatr}     {\mathrm{d}N_\mathrm{tracklets}/\mathrm{d}\eta}
\newcommand{\dNdetar}[1]   {\mathrm{d}N_\mathrm{ch}/\mathrm{d}\eta\left.\right|_{|\eta|<#1}}
\newcommand{\lum}          {\, \mbox{${\rm cm}^{-2} {\rm s}^{-1}$}}
%\newcommand{\barn}         {\, \mbox{${\rm barn}$}}
\newcommand{\m}            {\, \mbox{${\rm m}$}}
\newcommand{\ncls}         {\ensuremath{N_{cls}}}
\newcommand{\nsigma}       {\ensuremath{n\sigma}}
\newcommand{\dcaxy}        {\ensuremath{{\rm DCA}_{xy}} }
\newcommand{\dcaz}         {\ensuremath{{\rm DCA}_{z}} }
\newcommand{\EcrossB}      {E$\times$B}%{\ensuremath{{\rm E}\times{\rm B}}}
\newcommand{\bb}           {Bethe-Bloch}
\newcommand{\s}            {\ensuremath{\sqrt{s}}}
\newcommand{\pt}           {\ensuremath{p_{\mathrm{T}}}}
\newcommand{\pts}           {\ensuremath{p_{\rm T}} }
\newcommand{\hlab}         {\ensuremath{\eta_{\rm lab}}}
\newcommand{\ynn}         {\ensuremath{y_{\rm NN}}}
\newcommand{\ycms}         {\ensuremath{y_{\rm CMS}}}
\newcommand{\ylab}         {\ensuremath{y_{\rm lab}}}
\newcommand{\ppi}          {\ensuremath{{\rm p}/\pi}}
\newcommand{\kpi}          {\ensuremath{{\rm K}/\pi}}
\newcommand{\lpi}          {\ensuremath{{\rm \Lambda}/\pi}}
%\newcommand{\ppi}          {\ensuremath{(\pi^+ + \pi^-)/({\rm K}^+ + {\rm K}^-)}}
%\newcommand{\kpi}          {\ensuremath{({\rm p} + {\rm \bar p})/({\rm K}^+ + {\rm K}^-)}}
\newcommand{\mt}           {\ensuremath{m_{\rm T}}}
\newcommand{\snn}          {\ensuremath{\sqrt{s_{\rm NN}}}}
\newcommand{\snnbf}        {\ensuremath{\mathbf{{\sqrt{s_{\mathbf NN}}}}}}
\newcommand{\sonly}        {\ensuremath{\sqrt{s}}}
\newcommand{\Npart}        {\ensuremath{N_\mathrm{part}}}
\newcommand{\avNpart}      {\ensuremath{\langle N_\mathrm{part} \rangle}}
\newcommand{\avNpartdata}  {\ensuremath{\langle N_\mathrm{part}^{\rm data} \rangle}}
\newcommand{\Ncoll}        {\ensuremath{N_\mathrm{coll}}}
\newcommand{\Dnpart}       {\ensuremath{D\left(\Npart\right)}}
\newcommand{\DnpartExp}    {\ensuremath{D_{\rm exp}\left(\Npart\right)}}
\newcommand{\dNdetapt}     {\ensuremath{\dNdeta\,/\left(0.5\Npart\right)}}
\newcommand{\dNdetaptr}[1] {\ensuremath{\dNdetar{#1}\,/\left(0.5\Npart\right)}}
\newcommand{\dNdetape}     {\left(\ensuremath{\dNdeta\right)/\left(\avNpart/2\right)}}
\newcommand{\dNdetaper}[1] {\ensuremath{\dNdetar{#1}\,/\left(\avNpart/2\right)}}
\newcommand{\dndydpt}      {\ensuremath{{\rm d}^2N/({\rm d}y {\rm d}p_{\rm t})}}
\newcommand{\abs}[1]       {\ensuremath{\left|#1\right|}}
\newcommand{\signn}        {\ensuremath{\sigma^{\rm inel.}_{\rm NN}}}
\newcommand{\vz}           {\ensuremath{V_{z}}}
\newcommand{\Tfo}          {\ensuremath{{T}_{\rm kin}}}
\newcommand{\Tch}          {\ensuremath{{T}_{\rm ch}}}
\newcommand{\bT}           {\ensuremath{\beta_{\rm T}}}
\newcommand{\avbT}         {\ensuremath{\left< \beta_{\rm T}\right>}}
\newcommand{\avpT}         {\ensuremath{\left< \pt \right>}\xspace}
\newcommand{\muB}          {\ensuremath{\mu_{B}}}
\newcommand{\stat}         {({\it stat.})}
\newcommand{\syst}         {({\it sys.})}
\newcommand{\Fig}[1]       {Fig.~\ref{#1}}
\newcommand{\Figure}[1]    {Figure~\ref{#1}}
\newcommand{\Ref}[1]       {Ref.~\cite{#1}}
\newcommand{\green}[1]     {\textcolor{green}{#1}}
\newcommand{\blue}[1]      {\textcolor{blue}{#1}}
\newcommand{\red}[1]       {\textcolor{red}{#1}}
\newcommand{\white}[1]     {\textcolor{white}{#1}}
\newcommand{\gevc}         {\ensuremath{{\rm GeV}/c}}
\newcommand{\mevc}         {\ensuremath{{\rm MeV}/c}}
\newcommand{\gevcs}         {\ensuremath{{\rm GeV}/c} }
\newcommand{\mevcs}         {\ensuremath{{\rm MeV}/c} }
\newcommand{\avg}[1]       {\ensuremath{\left\langle#1\right\rangle}}

\newcommand {\dEdx}      {d\textit{E}/d\textit{x}\xspace}
\newcommand {\Zvtx}   {\ensuremath{Z_\mathrm{vtx}}\xspace}
\newcommand {\pT}   {\pt}

\newcommand {\proton}     		{\ensuremath{p}}
\newcommand {\electron}   		{\Pe}
\newcommand {\pion}    	  		{\ensuremath{\pi}}
\newcommand {\kaon}       		{\ensuremath{K}}
\newcommand {\KTopi}      		{\kaon/\pion}
\newcommand {\pTopi}      		{\proton/\pion}
\newcommand {\KzeroShort} 		{\PKzS}
\newcommand {\lambdaBaryon} 	{\PGL}
\newcommand {\antiLambdaBaryon} {\PAGL}
\newcommand {\gammaPhoton} 		{\PGg}
\newcommand {\Nevt}      {\ensuremath{N_\mathrm{evt}}\xspace}
\newcommand {\NevtMB}  {\ensuremath{N_\mathrm{evt|MB}}\xspace}
\newcommand {\NevtMBVTX}  {\ensuremath{N_\mathrm{evt|(MB\, \&\, Vtx)}}\xspace}
\newcommand {\NevtMBVTXZ}  {\ensuremath{N_\mathrm{evt|(MB\, \& Vtx\, \&\, \textit{Z}_{vtx})}}\xspace}
\newcommand {\NevtINEL}  {\ensuremath{N_\mathrm{evt}(\textsc{inel})}\xspace}
\newcommand {\fPrim}       {\ensuremath{f_{\mathrm{prim}}}\xspace}
\newcommand {\NPrim}     {\ensuremath{N_{\mathrm{prim}}}\xspace}
\newcommand {\NSec}      {\ensuremath{N_{\mathrm{sec}}}\xspace}
\newcommand {\NPrimTilde}     {\ensuremath{\widetilde{N}_{\mathrm{prim}}}\xspace}
\newcommand {\NSecTilde}      {\ensuremath{\widetilde{N}_{\mathrm{sec}}}\xspace}
\newcommand {\NTilde}      {\ensuremath{\widetilde{N}}\xspace}

\newcommand{\pPiplus}{\ensuremath{{\pi}^{+}}\xspace}
\newcommand{\pPiminus}{\ensuremath{{\pi}^{-}}\xspace}
\newcommand{\sPi}{\ensuremath{{\pi}}\xspace}
\newcommand{\pKplus}{\ensuremath{{\rm K}^{+}}\xspace}
\newcommand{\pKminus}{\ensuremath{{\rm K}^{-}}\xspace}
\newcommand{\sProton}{\ensuremath{\rm p}\xspace}
\newcommand{\pProton}{\ensuremath{\rm p}\xspace}
\newcommand{\apProton}{\ensuremath{\overline{\rm p}}\xspace}
\newcommand{\sPr}{\ensuremath{\rm p}\xspace}
\newcommand{\sKzero}{\ensuremath{2{\rm K}^{0}_{S}}\xspace}
\newcommand{\pKzero}{\ensuremath{{\rm K}^{0}_{S}}\xspace}
\newcommand{\sLambda}{\ensuremath{\Lambda}\xspace}
\newcommand{\pLambda}{\ensuremath{\Lambda}\xspace}

\newcommand{\LtoKzero}{\ensuremath{\Lambda}/\ensuremath{{\rm K}^{0}_{S}}\xspace}
\newcommand{\apLambda}{\ensuremath{\overline{\Lambda}}\xspace}
\newcommand{\sXi}{\ensuremath{\Xi}\xspace}
\newcommand{\pXi}{\ensuremath{\Xi^{-}}\xspace}
\newcommand{\apXi}{\ensuremath{\overline{\Xi}^{+}}\xspace}
\newcommand{\sOmega}{\ensuremath{\Omega}\xspace}
\newcommand{\pOmega}{\ensuremath{\Omega^{-}}\xspace}
\newcommand{\apOmega}{\ensuremath{\overline{\Omega}^{+}}\xspace}

\newcommand{\betaT}{\ensuremath{\langle \beta_{T}\rangle}\xspace}
\newcommand{\Tkin}{\ensuremath{T_{kin}}\xspace}

\renewcommand{\labelitemi} {$-$}
%==========================================================%
%%% inline warnings for internal discussion 
%\newcommand{\warn}[1]      {\textbf{\textcolor{red}{[#1]}}}
\newcommand{\warn}[1]      {{\small\textbf{\textcolor{red}{(!\footnote{\textbf{(!)}~#1})}}}}
%\newcommand{\warn}[1]      {{\small\textbf{(!\footnote{\textbf{(!)}~#1})}}\marginpar{\textbf{---}}}
\newcommand{\todo}[1]      {\textbf{\textcolor{red}{[TODO: #1]}}}
%%% fake numbers
\newcommand{\fake}[1]      {\textbf{\textcolor{red}{#1}}}
%\newcommand{\fake}[1]      {#1}
\newcommand{\final}[1]     {\textbf{\textcolor{blue}{#1}}}
\newcommand{\prelim}[1]    {\textbf{\textcolor{magenta}{#1}}}
\renewcommand{\mod}[1]       {\textbf{\textcolor{red}{#1}}}



\begin{document}
\begin{center}
{\Large {\bf Testing coalescence and statistical-thermal production scenarios for (anti-)(hyper-)nuclei at LHC energies with recent and future Run 3 and 4 data}}

\medskip

F. Bellini and A. Kalweit

\medskip

26th of February 2018
\end{center}

\bigskip

%
% Introduction 
%
%\listoffigures
%\listoftables
%\newpage

\begin{abstract}
(Anti-)(hyper-)nuclei are unique probes of the medium created in proton-proton, proton-Pb, and Pb--Pb collisions at LHC energies. At LHC energies, their production is typically discussed within the framework of coalescence and thermal-statistical production models. While it is often argued that both approaches are not distinguishable, we present a detailed study of both theories which reveals largely different predictions between the two approach for the production of 3He and hyper-tritons. Confronting our results with recent ALICE measurements, the coalescence approach is found to provide a correct description of the data only in small systems such as pp collisions, while it fails for central Pb--Pb collisions. The thermal-statistical model on the other hand is in agreement with results in central Pb--Pb collisions even though such fragile objects should be destroyed in hadronic interactions after the chemical freeze-out of the system. Our finding thus indicate the existence of a novel production mechanism for these objects.
\end{abstract}


%%Table of contents - to be removed for submission%%
\tableofcontents

\section{Introduction} 
The formation of light anti- and hyper- nuclei in highly energetic proton-proton, proton-nucleus and nucleus-nucleus collisions provides unique observables for the study of the system created in these collisions. 
In this context, nuclei and hyper-nuclei are special objects with respect to non-composite hadrons, because their size is comparable to a fraction or the whole system created in the collision \cite{}. 
As quantum-mechanical objects, their size is typically defined as the rms of their wave function, which ranges from \textcolor{red}{xx fm for the 3He up to yy fm for the hyper-triton}. 
Halo nuclei as 6He would be ideal for such studies, but they remain out of the experimental reach in high-energy experiments. 

Surprisingly, thermal-statistical models have been successful at describing not only light-flavour particle production, but also that of light (anti-)(hyper-)nuclei across a wide range of energies in nucleus-nucleus collisions \cite{Andronic:2017, Andronic:2010qu}. 
In this approach, particles are produced from a fireball in thermal and kinetic equilibrium with temperatures of the order of $T_{chem}$ = 156 MeV (near the temperature at the QCD phase transition boundary, as predicted by lattice QCD calculations \cite{T from lattice}. Particle abundances are fixed at chemical freeze-out, when inelastic collisions cease. Further elastic and pseudo-elastic collisions occur among the components of the expanding fireball, that can affect the spectral shapes and the measurable yields of short-lived (strongly decaying) hadronic resonances. Once the particle density of the system is so low that the mean free path for elastic collisions is larger than the size of the system, the fireball freezes-out kinetically. This is seen to occur when the system has reached temperatures of the order of $T_{kin} \approx$ 90 MeV. 
In such a dense and hot environment, composite objects with binding energies that are small with respect to the temperature of the system, appear as ``fragile'' objects. For instance, the binding energy of the deuteron is $E_{B, d}$ = 2.2 MeV $\ll T_{chem}, T_{kin}$.
As a matter of fact, the cross-section for pion-induced deuteron breakup is significantly larger than the typical (pseudo)-elastic cross-sections for the re-scattering of hadronic resonance decay products (check statement and refer to Juergen's QM proceedings) \cite{Garcilazo:1982yc, Bass:1998ca}. Similarly, the elastic cross-section which drives the deuteron spectra to kinetic equilibration in central heavy-ion collisions \cite{Acharya:2017dmc} is \textcolor{red}{smaller than the breakup cross-section} \cite{Schukraft:2017nbn}.   
Based on this, the deuterons produced at chemical freeze-out would be expected not to survive the hadronic phase of the medium expansion, yet they're production is measured to be consistent to the predictions from statistical-thermal models and they develop also a non-zero elliptic flow which is consistent with a common radial expansion together with the non-composite hadrons \cite{Acharya:2017dmc}. \textcolor{Do similar estimates as Karel in Frascati}. In addition, it was recently shown that the assumption of realistic eigenvolumina for light nuclei would lead to instabilities of the statistical/thermal model predictions \cite{Vovchenko:2016mwg}.
Several solutions have been proposed to solve this ``light (anti-)nuclei puzzle'': (a.) a sudden freeze-out at the QGP-hadron phase boundary, (b.) the thermal production of these objects a compact quark bags \cite{Andronic:2017}, and (c.) the coincidence of coalescence mechanism with that of thermal production \cite{Scheibl:1998tk}.
Data from rescattering of short-lived hadronic resonances indicate that the system undergoes a long-lasting hadronic phase before decoupling \cite{}, thus strongly disfavouring hypothesis (a.). 
While hypothesis (b.) cannot presently be tested beyond the agreement of measured (anti-)nuclei production yields with statistical-thermal model predictions, hypothesis (c.) is scrutinised in the present work.

To this purpose, we compare to models the existing data from the Large Hadron Collider which for the first time allow for the systematic study of the light (anti-)(hyper-)nuclei production as a function of the system and object size. 
In the nucleon-coalescence approach, nuclei are formed at kinetic freeze-out by coalescence of nucleons that are nearby in space and have similar velocities. 

- coalescence approaches

- first time we see a plot where we compare directly the two main approaches with data

- B2 as the observable for Run 3 and 4 where to learn the most 

- content of the paper



\section{Coalescence approach}

\subsection{Source volume}

\section{Statistical-thermal approach and Blast-wave}

\section{Comparison with experimental data}

\subsection{(Anti-)nuclei with $A$ = 2, 3, 4}

\subsection{(Anti-)hyper-nuclei}

\section{Projections for the LHC Run 3 and 4}

\section{Summary and conclusions}

We conclude that (c.) appears unlikely, thus leaving (b.) as a viable option, at least with our present knowledge of the hypertriton size.

%%%%% acknowledgements
\newenvironment{acknowledgement}{\relax}{\relax}
\begin{acknowledgement}
\section*{Acknowledgements}

The authors would like to thank themselves for the auto-critics.

\end{acknowledgement}




\bibliographystyle{utphys} 	
\bibliography{NucleiB2}




\end{document}




%%
%% End of file.
